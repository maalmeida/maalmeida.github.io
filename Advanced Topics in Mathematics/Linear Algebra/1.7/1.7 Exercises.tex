\documentclass[11pt,answers]{exam} %noanswers
\usepackage{multicol}
\usepackage[usenames]{color} %used for font color
\usepackage{amssymb} %maths
\usepackage{amsmath} %maths
\usepackage[utf8]{inputenc} %useful to type directly diacritic characters
\usepackage{siunitx}
\usepackage{mathtools}
\usepackage[version=4]{mhchem}
\newcommand{\qtty}[2]{\left(\qty{#1}{#2}\right)}
\sisetup{display-per-mode = symbol }
\sisetup{inter-unit-product = \ensuremath { { } \cdot { } } }
\usepackage[letterpaper, total={7in, 10in}]{geometry}
\usepackage{graphicx} % Required for inserting images
\renewcommand{\epsilon}{\varepsilon}
\renewcommand{\vec}[1]{\mathbf{#1}}
\newcommand{\mat}[1]{\begin{bmatrix*}[r] #1 \end{bmatrix*}}

\begin{document}
\flushleft
Name:\ \makebox[4in]{\hrulefill} \hspace{1in} Date:\ \makebox[1in]{\hrulefill}
\begin{center}
    \textbf{\S 1.7 Linear Independence}
\end{center}


\begin{questions}
    \uplevel{Determine if the  vectors are linearly independent. Justify each answer.}
    \vspace{0.25cm}
    \begin{multicols}{4}
        \question $ \displaystyle \smash{\begin{bmatrix*}[r] 5 \\ 1 \\ 0 \end{bmatrix*}, \begin{bmatrix*}[r] 7 \\ 2 \\ -6 \end{bmatrix*}, \begin{bmatrix*}[r] -2 \\ -1 \\ 6 \end{bmatrix*}} $
        % \begin{solution}
        %     fdsfds
        % \end{solution}
        \question $ \displaystyle \smash{\begin{bmatrix*}[r] 0 \\ 0 \\ 0 \end{bmatrix*}, \begin{bmatrix*}[r] 0 \\ 5 \\ -8 \end{bmatrix*}, \begin{bmatrix*}[r] -3 \\ 4 \\ 1 \end{bmatrix*}} $
        % \begin{solution}
        %     This should be <0,0,2> which makes it linearly independent. However, it's nice to have something with <0,0,0> which \emph{automatically} makes it linearly dependent since nv1+0v2+0v2 = 0 vector for any n
        % \end{solution}
        \question $ \displaystyle \smash{\begin{bmatrix*}[r] 1 \\ -3 \end{bmatrix*}, \begin{bmatrix*}[r] -3 \\ 6 \end{bmatrix*}} $
        \question $ \displaystyle \smash{\begin{bmatrix*}[r] -1 \\ 4 \end{bmatrix*}, \begin{bmatrix*}[r] -2 \\ 8 \end{bmatrix*}}$
    \end{multicols}

    \uplevel{Determine if the columns of the matrix form a linearly independent set. Justify each answer}
    \vspace{0.5cm}
    \begin{multicols}{4}
        \question $ \displaystyle  \smash{\begin{bmatrix*}[r] 0 & -8 & 5 \\ 3 & -7 & 4 \\ -1 & 5 & -4 \\ 1 & -3 & 2 \end{bmatrix*}} $ 
        \newline \vspace{2.0cm} text
        \question $ \displaystyle  \smash{\begin{bmatrix*}[r] -4 & -3 & 0 \\ 0 & -1 & 4 \\ 1 & 0 & 3 \\ 5 & 4 & 6 \end{bmatrix*}} $
        \question $ \displaystyle  \smash{\begin{bmatrix*}[r] 1 & 4 & -3 & 0 \\ -2 & -7 & 5 & 1 \\ -4 & -5 & 7 & 5 \end{bmatrix*}} $
        \question $ \displaystyle  \smash{\begin{bmatrix*}[r]1 & -3 & 3 & -2 \\ -3 & 7 & -1 & 2 \\ 0 & 1 & -4 & 3 \end{bmatrix*}} $
    \end{multicols}
    \vspace{0.5cm}
    \begin{multicols}{4}
        \rule{1cm}{0.15mm} \columnbreak \rule{1cm}{0.15mm} \columnbreak \rule{1cm}{0.15mm} \columnbreak \rule{1cm}{0.15mm}
    \end{multicols}

    \vspace{0.25cm}
    \uplevel{(a) For what values of $h$ is $\vec{v}_3$ in $\textrm{Span}\{\vec{v}_1, \vec{v}_2\}$ (b) For what values of $h$ is $\{\vec{v}_2, \vec{v}_2, \vec{v}_3\}$ linearly \emph{dependent}? Justify each answer. }
    \question $ \displaystyle \vec{v}_1 = \begin{bmatrix*}[r] 1 \\ -3 \\ 2 \end{bmatrix*}, \vec{v}_2 = \begin{bmatrix*}[r] -3 \\ 10 \\ -6 \end{bmatrix*}, \vec{v}_3 = \begin{bmatrix*}[r] 2 \\ -7 \\ h \end{bmatrix*}$
    \question $ \displaystyle \vec{v}_1 = \begin{bmatrix*}[r] 1 \\ -5 \\ -3 \end{bmatrix*}, \vec{v}_2 = \begin{bmatrix*}[r] -2 \\ 10 \\ 6 \end{bmatrix*}, \vec{v}_3 = \begin{bmatrix*}[r] 2 \\ -10 \\ h \end{bmatrix*}$

    \vspace{0.25cm}
    \uplevel{Find the value(s) of $h$ for which the vectors are linearly \emph{dependent}. Justify each answer.}
    \question $ \displaystyle \vec{v}_1 = \begin{bmatrix*}[r] 1 \\ -1 \\ 4 \end{bmatrix*}, \vec{v}_2 = \begin{bmatrix*}[r] 3 \\ -5 \\ 7 \end{bmatrix*}, \vec{v}_3 = \begin{bmatrix*}[r] -1 \\ 5 \\ h \end{bmatrix*}$
    \question $ \displaystyle \vec{v}_1 = \begin{bmatrix*}[r] 2 \\ -4 \\ 1 \end{bmatrix*}, \vec{v}_2 = \begin{bmatrix*}[r] -6 \\ 7 \\ -3 \end{bmatrix*}, \vec{v}_3 = \begin{bmatrix*}[r]8 \\ h \\ 4 \end{bmatrix*}$
    \question $ \displaystyle \vec{v}_1 = \begin{bmatrix*}[r] 1 \\ 5 \\ -3 \end{bmatrix*}, \vec{v}_2 = \begin{bmatrix*}[r] -2 \\ -9 \\ 6 \end{bmatrix*}, \vec{v}_3 = \begin{bmatrix*}[r] 3 \\ h \\ -9 \end{bmatrix*}$
    \question $ \displaystyle \vec{v}_1 = \begin{bmatrix*}[r] 1 \\ -1 \\ 3 \end{bmatrix*}, \vec{v}_2 = \begin{bmatrix*}[r] -5 \\ 7 \\ 8 \end{bmatrix*}, \vec{v}_3 = \begin{bmatrix*}[r] 1 \\ 1 \\ h \end{bmatrix*}$

    \uplevel{Determine by inspection whether the following sets of vectors are linearly \emph{independent}. Justify each answer.}
    \begin{multicols}{2}
        \question $\mat{ 5 \\ 1}, \mat{2 \\ 8}, \mat{1 \\ 3}, \mat{-1 \\ 7}$
        \question $\mat{4 \\ -2 \\ 6}, \mat{6 \\ -3 \\ 9}$    
        \question $\mat{3 \\ 5 \\ 1}, \mat{0 \\ 0 \\ 0}, \mat{-6 \\ 5 \\ 4}$
        \question $\mat{4 \\ 4}, \mat{-1 \\ 3}, \mat{2 \\ 5}, \mat{8 \\ 1}$
        \question $\mat{-8 \\ 12 \\ -4}, \mat{2 \\ -3 \\ -1}$
        \question $\mat{1 \\ 4 \\ -7}, \mat{-2 \\ 5 \\ 3}, \mat{0 \\ 0 \\ 0}$
    \end{multicols}
    
   

    \uplevel{Mark each statement True or False (\textbf{T/F}). Justify each answer on the basis of a careful reading of the text.}
    \question (\textbf{T/F}) The columns of a matrix $A$ are linearly independent if the equation $A\vec{x}=\vec{0}$ has the trivial solution.
    \question (\textbf{T/F}) Two vectors are linearly independent if and only if they lie on a line through the origin.
    \question (\textbf{T/F}) If $S$ is a linearly indpendent set, then each vector is a linear combination of the other vectors in $S$.
    \question (\textbf{T/F}) If a set contains fewer vectors than there are entries in the vectors, then the set is linearly independent.
    \question (\textbf{T/F}) The columns of any $4\times5$ matrix are linearly independent.
    \question (\textbf{T/F}) If $\vec{x}$ and $\vec{y}$ are linearly independent, and if $\vec{z}$ is in $\textrm{Span}\{\vec{x}, \vec{y}\}$, then $\{ \vec{x}, \vec{y}, \vec{z} \}$ linearly dependent.
    \question (\textbf{T/F}) If $\vec{x}$ and $\vec{y}$ are linearly independent and if $\{ \vec{x}, \vec{y}, \vec{z} \}$ is linearly dependent, then $\vec{z}$ is in $\textrm{Span}\{\vec{x},\vec{y}\}$.
    \question (\textbf{T/F}) If a set in $\mathbb{R}^n$ is linearly independent, then the set contains more vectors than there are entires in each vector.

    \uplevel{Describe the possible echelon forms of the matrix. Use the notation of Example 1 in Section 1.2.}
    \question $A$ is a $3 \times 3$ matrix with linearly independent columns.
    \question $A$ is a $2 \times 2$ matrix with linearly dependent columns.
    \question $A$ is a $4 \times 2$ matrix, $A = \mat{ \vec{a}_1 & \vec{a}_2 }$, and $\vec{a}_2$ is a not a multiple of $\vec{a}_1$. 
    \question $A$ is a $4 \times 3$ matrix, $A = \mat{ \vec{a}_1 & \vec{a}_2 & \vec{v}_3}$ such that $\{ \vec{a}_1, \vec{a}_2\}$ is linearly independent and $\vec{a}_3$ is not in $\textrm{Span}\{\vec{a}_1, \vec{a}_2 \}$

    \vspace{0.25cm}
    \question How many pivot columns must a $7 \times 5$ matrix have if its columns are linearly independent? Why?
    \question How many pivot columns must a $5 \times 7$ matrix have if its columns span $\mathbb{R^5}$? Why?
    \question Construct $3 \times 2$ matrices $A$ and $B$ such that $A\vec{x}=\vec{0}$ has only the trivial solution and $B\vec{x}=\vec{0}$ has a nontrivial solution.
    \question
    \begin{parts}
        \part Fill in the blank in the following statement: ``If $A$ is an $m \times n$ matrix, then the columns of $A$ are linearly independent if and only if $A$ has \rule{1cm}{0.15mm} pivot columns."
        \part Explain why the statement in (a) is true.
    \end{parts}

    \vspace{5 cm}
    \uplevel{The following exercises should be  solved \emph{without performing row operations}.}
    \question Given $A = \mat{2 & 3 & 5 \\ -5 & 1 & -4 \\ -3 & -1 & -4 \\ 1 & 0 & 1}$, observe that the third column is the sum of the first two columns. Find a nontrivial solution to $A\vec{x}=\vec{0}$.
    \question Given $A = \mat{4 & 1 & 6 \\ -7 & 5 &  3 \\ 9 & -3 & 3}$, observe that the first column plus twice the second column equals the third column. Find a nontrivial solution of $A\vec{x}=\vec{0}$.
    \question this is a question
    \begin{solution} Questions \ref{question@1}
        This is the solution to question 1
    \end{solution}
    \begin{solution} Question \ref{question@2}
        This is the solution to question 2
    \end{solution}
    \question $f_x$
\end{questions}

\flushleft
\newpage
Describe the possible echelon forms of the matrix. Use the notation of Example 1 in 

\[
\begin{bmatrix} 5 \\ 1 \\ 0 \end{bmatrix}
\begin{bmatrix} 7 \\ 2 \\ -6 \end{bmatrix}, 
\begin{bmatrix} -2 \\ -1 \\ 6 \end{bmatrix}
\]

\[
\begin{bmatrix} 0 \\ 0 \\ 2 \end{bmatrix}
\begin{bmatrix} 0 \\ 5 \\ -8 \end{bmatrix}, 
\begin{bmatrix} -3 \\ 4 \\ 1 \end{bmatrix}
\]


\begin{equation*}
    A\mathbf{x}= \begin{bmatrix*}[r] \mathbf{a}_1 & \mathbf{a}_2 & \ldots & \mathbf{a}_n \end{bmatrix*} 
    \begin{bmatrix*}[c]
        x_1 \\ \vdots \\ x_n        
    \end{bmatrix*}
    = x_1 \mathbf{a}_1 + x_2 \mathbf{a}_2 + \ldots + x_n \mathbf{a}_n
\end{equation*}

\textbf{Multiply:}\
\begin{flalign*}
    \begin{bmatrix*}[r]
    1 & 2 & -1 \\
    0 & -5 & 3
    \end{bmatrix*}
    \begin{bmatrix*}[r] 4 \\ 3 \\ 7 \end{bmatrix*}
\end{flalign*}

\begin{flalign*}
    \begin{bmatrix*}[r]
    2 & -3 \\
    8 & 0 \\
    -5 & 2
    \end{bmatrix*}
    \begin{bmatrix*}[r] 4 \\ 7 \end{bmatrix*}
\end{flalign*}

========= \newpage
Determine if the following sets of vectors are linearly independent columns of the following matrices form a linearly independent set. Justify each answer.
\[ \begin{bmatrix*}[r] 5 \\ 1 \\ 0 \end{bmatrix*}, 
\begin{bmatrix*}[r] 7 \\ 2 \\ -6 \end{bmatrix*}, 
\begin{bmatrix*}[r] -2 \\ -1 \\ 6 \end{bmatrix*} \]




\end{document}
