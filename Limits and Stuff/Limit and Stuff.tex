\documentclass[addpoints,solutions]{exam}
\usepackage{multicol}
\usepackage{graphics,color,graphicx}
\usepackage{bm}
\usepackage[letterpaper, margin=1in]{geometry}
%\usepackage{pgfplots}
\usepackage{capt-of}
\usepackage{hyperref}
\usepackage{subfigure}
%\pgfplotsset{width=15cm, height=5cm,compat=1.9}
\newcommand{\ox}[1]{ \noindent\fbox{\parbox{\columnwidth}{#1 }}}
\newcommand{\e}{\epsilon_0}
\newcommand{\f}[2]{\frac{#1}{#2}}
\newcommand{\pderiv}[2]{\frac{\partial #1}{\partial #2}}
\newcommand{\bd}{\boldsymbol}

\begin{document}

\flushleft \textbf{Directions: } These are my directions.

\begin{questions}

\question This problem taken from this \href{www.google.com}{source}. Consider the function $y=f \left(x\right)$ displayed below. 
.
\begin{figure}[h]
\begin{center}
\includegraphics[width=5cm]{g25.png}
\caption{Graph of $y=f\left(x\right)$}
\label{default}
\end{center}
\end{figure}



\begin{parts}
\part At which value(s) of $x$ does the function $f$ have a removable discontinuity? If $f$ has no removable discontinuities, write "NONE". \answerline
\part At which value(s) of $x$ does the function $f$ have a jump discontinuity? If $f$ has no removable discontinuities, write "NONE". \answerline
\part At which value(s) of $x$ does the function $f$ have an infinite discontinuity? If $f$ has no removable discontinuities, write "NONE". \answerline
\part Fill in the blank with either the word ``right" or the word "left". At the point where $x=1$, the function $f$ is \fillin \ continuous.
\part Fill in the blank with either the word ``right" or the word "left". At the point where $x=5$, the function $f$ is \fillin \ continuous.
\end{parts}
\end{questions}




%\flushleft \textbf{Quiz: } (Topics 1.1 - 1.3) Rate of Change in Functions \hspace{2cm} Name: \rule{5cm}{0.1mm}
%\noindent\hrulefill
%
%\flushleft
%\textbf{Directions: NO CALCULATORS.} Use the graph of the following function to answer the following:
%
%\begin{figure}[htbp]
%\begin{center}
%\includegraphics[scale=0.25]{IMG_01}
%\caption{Graph of $g$}
%\label{default}
%\end{center}
%\end{figure}
%
%\begin{questions}
%\question The graph of a function $f$ is shown in the figure above for the interval $-6 \leq x \leq 6$. Use the graph of $f$ to answer the following.
%\begin{multicols}{2}
%\begin{parts}
%\part On what open interval(s) is $f$ increasing?
%\part On what interval(s) is $f$ decreasing?
%\end{parts}
%\end{multicols}
%\end{questions}
%
%\hrule
%\vspace{1cm}
%This is a test $\int_0^\infty f\left(x\right)$ \\
%\vspace{1cm}
%dfsfsf
%
%\begin{tikzpicture}
%\begin{axis}[
%  %axis lines = left,
%  axis lines = middle,
%  axis line style={-stealth,shorten >=-3mm},
%  width=8cm,
%  height=8cm,
%  xlabel ={$x$},
%  ylabel ={$y$}, 
%  xmin=-2, xmax=9,
%  ymin=-1, ymax=11,
%  xtick={-2,-1,...,9},
%  ytick={-1,0,...,11},
%  xmajorgrids=true, ymajorgrids=true,
%  minor grid style = {very thin, gray},
%  major grid style={solid, thick, black},
%  xminorgrids=true, yminorgrids=true, minor tick num = 1,
%  %grid style = dashed,
%  minor x tick num = 9,
%  minor y tick num = 4,
%  %minor tick style = {thin, gray},
%  %major tick style = {thick, black},
%  minor tick style={draw=none},
%  xticklabel style={/pgf/number format/.cd, fixed relative, precision=4, /tikz/.cd},
%  yticklabel style={/pgf/number format/.cd,fixed relative,precision=3}]
%  %style="yticklabel style={
%  %/pgf/number format/precision=5,
%  %/pgf/number format/fixed}",
%  %scaled y ticks=false]
%
%%\addplot[color=red]{exp(x)};
%%\addplot[color=blue, samples=100]{x^2};
%
%\addplot[color=purple, domain=-3:6, samples=1000, style=ultra thick]{5+sqrt(1.5^2-(x-4.5)^2)};
%\end{axis}
%\end{tikzpicture}
%
%\includegraphics{standalone}
%\includegraphics{IMG_02}
%\includegraphics{Passwwater}
%
%\begin{multicols}{2}
%\begin{figure}[h!]
%  \centering
%  \includegraphics{Passwwater}
%  \caption{Graph of $p$}
%\end{figure}
%
%\begin{figure}[h!]
%  \centering
%  \includegraphics{Passwwater}
%  \caption{Graph of $q$}
%\end{figure}
%\end{multicols}



\end{document}